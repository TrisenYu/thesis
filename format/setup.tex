%%%% contact via <kisfg@hotmail.com, haikureimu@hnu.edu.cn>
%%% 页边距定义
% 上边距 3cm
% 下边距 2.5cm
% 页眉到上边缘的距离 1.5cm
% 页脚到下边缘的距离 1.75cm
\setlength{\voffset}{0cm}
\setlength{\topmargin}{1.5cm-1in}
\setlength{\headheight}{13pt} % 页眉可以放得大一点
\setlength{\headsep}{1.5cm-12pt}
\setlength{\textheight}{\paperheight-5.5cm}
\setlength{\footskip}{0.75cm}

%%% 控制行距
\renewcommand{\baselinestretch}{1.5}
\setlength{\parindent}{2em}

%%% 定义字体
\newcommand{\song}{\songti}     % 宋体 rm
\newcommand{\hei}{\heiti}       % 黑体 bf 
\newcommand{\fs}{\fangsong}     % 仿宋体 tt
\newcommand{\kai}{\kaishu}      % 楷体 it
\newcommand{\li}{\lishu}        % 隶书


%%%% 字号换算规范
%%%% 中文常用字号来代替国际用点数/磅/英寸/pt。
%% 参考 https://www.runoob.com/w3cnote/px-pt-em-convert-table.html
\newcommand{\chuhao}{\fontsize{42pt}{42pt}\selectfont}      % 初号, 1.倍行距
\newcommand{\yihao}{\fontsize{26pt}{26pt}\selectfont}       % 一号, 1.倍行距
\newcommand{\xiaoyi}{\fontsize{24pt}{24pt}\selectfont}      % 小一, 1.倍行距
\newcommand{\erhao}{\fontsize{22pt}{22pt}\selectfont}       % 二号, 1.倍行距
\newcommand{\xiaoer}{\fontsize{18pt}{18pt}\selectfont}      % 小二, 单倍行距
\newcommand{\sanhao}{\fontsize{16pt}{16pt}\selectfont}      % 三号, 1.倍行距
\newcommand{\xiaosan}{\fontsize{15pt}{15pt}\selectfont}     % 小三, 1.倍行距
\newcommand{\sihao}{\fontsize{14pt}{14pt}\selectfont}       % 四号, 1.0倍行距
\newcommand{\xiaosi}{\fontsize{12pt}{12pt}\selectfont}      % 小四, 1.倍行距
\newcommand{\wuhao}{\fontsize{10pt}{10pt}\selectfont}       % 五号, 单倍行距
\newcommand{\xiaowu}{\fontsize{8pt}{8pt}\selectfont}        % 小五, 单倍行距
\CJKtilde

\setitemize{leftmargin=3em,itemsep=0em,partopsep=0em,parsep=0em,topsep=-0em}
\setenumerate{leftmargin=3em,itemsep=0em,partopsep=0em,parsep=0em,topsep=0em}
\makeatletter
    \renewcommand\normalsize{
        \@setfontsize\normalsize{12pt}{12pt}
        \setlength\abovedisplayskip{4pt}
        \setlength\abovedisplayshortskip{4pt}
        \setlength\belowdisplayskip{\abovedisplayskip}
        \setlength\belowdisplayshortskip{\abovedisplayshortskip}
        \let\@listi\@listI}
    \def\defaultfont{\renewcommand{\baselinestretch}{2}\normalsize\selectfont}
    \setlength{\baselineskip}{20pt}
    \renewcommand{\CJKglue}{\hskip 0.5pt plus \baselineskip}
\makeatother

%% 这个块不要动,否则会引起 hyperref 的跳转错误。
\def\functor{\relax}
\let\functor\addcontentsline
\gdef\addcontentsline{\phantomsection\functor} % 这样就能修改addcontentsline的功能,使其额外插入。

%%% 配置目录
\renewcommand*{\contentsname}{\hfill 目\qquad\quad录 \hfill}
\setcounter{secnumdepth}{3}     % 编号深度 3
\setcounter{tocdepth}{3}        % 目录项深度为 3
\renewcommand{\thefigure}{\arabic{figure}}
\renewcommand{\thetable}{\arabic{table}}
\renewcommand{\theequation}{\arabic{equation}} % bold equation numbers
%% 目录项中的基本配件。
\renewcommand{\cftdot}{$...$}   % 点的样式。
\renewcommand{\cftdotsep}{1.5}  % TODO: 这里的单位是什么?

% 调整索引与文献名称之间的水平间距
\renewcommand{\biblabelsep}{6pt}

% section 从 1 开始计数
\renewcommand{\thesection}{\arabic{section}}
% section 格式配置
\titlecontents{section}[0em]    % 标题级别 标题位置 (左间距,理解为缩进)
    {\xiaosi\hei\vspace{0.5em}} % 标题样式
    {\textbf{\thecontentslabel}\hspace{0.5em}} 
    {} % 标题格式 (设置标题的整体格式,如字体尺寸、粗细、与上一个标题的垂直距离等,可空置)
    {\titlerule*[10pt]{\cftdot}\xiaosi\contentspage} 
    % 指引线与页码 (设置标题与页码之间的指引线样式以及页码的格式,该参数如果空置,标题将无指引线和页码)

% subsection 格式配置
\titlecontents{subsection}[2em]
    {\xiaosi\song\vspace{0.5em}}
    {\thecontentslabel\hspace{0.5em}}
    {}
    {\titlerule*[10pt]{\cftdot}\xiaosi\contentspage}

% subsubsection 格式配置
\titlecontents{subsubsection}[4em]
    {\xiaosi\song\vspace{0.5em}}
    {\thecontentslabel\hspace{0.5em}}
    {}
    {\titlerule*[10pt]{\cftdot}\xiaosi\contentspage}

%%% 图表目录以及格式定义。
\renewcommand*{\listfigurename}{\hfill 插图索引 \hfill}
\renewcommand*{\listtablename}{\hfill 附表索引 \hfill}
\setcounter{lofdepth}{1}
% figure index 的格式配置
\titlecontents{figure}[0em]
    {\xiaosi\song\vspace{0.5em}} %
    {图\hspace{0.5em}\textnormal{\thecontentslabel}\hspace{0.5em}}{} %
    {\titlerule*[10pt]{\cftdot}\xiaosi\contentspage}

% table index 的格式配置
\titlecontents{table}[0em]
    {\xiaosi\song\vspace{0.5em}} %
    {表\hspace{0.5em}\textnormal{\thecontentslabel}\hspace{0.5em}}{} %
    {\titlerule*[10pt]{\cftdot}\xiaosi\contentspage}

%% 设置插图目录条目的字体格式
% 直接重定义字体即可。
% https://tex.stackexchange.com/a/372825
% https://tex.stackexchange.com/a/192450
\DeclareCaptionFont{chineseBoldSongti}{\bfseries\songti\wuhao}
\DeclareCaptionFont{chineseWuHaoSongti}{\songti\wuhao}
\DeclareCaptionFont{chinesexiaowuSongti}{\songti\xiaowu}
\DeclareCaptionFont{chineseBoldHeiti}{\bfseries\heiti\wuhao}
\DeclareCaptionFont{chineseWuhao}{\wuhao}
\DeclareCaptionLabelFormat{figLabel}{#1~#2\hspace{0.5em}}
\captionsetup[figure]{
    name={\textbf{\song\wuhao{图}}},
    labelsep=space, % 去掉冒号
    labelformat=figLabel,
    font={chineseWuhao},          
    labelfont={bf},         
    textfont={chineseBoldSongti},   
    skip=0.5em,
}
\captionsetup[subfigure]{
    font={chineseWuhao},
    labelfont={chineseWuHaoSongti},
    textfont={chineseWuHaoSongti}
}
\DeclareCaptionLabelFormat{tabLabel}{#1~#2~}
\captionsetup[table]{
    name={\textbf{\hei\wuhao{表}}},
    labelsep=space, % 去掉冒号
    labelformat=tabLabel,
    font={chineseWuhao},
    labelfont={bf},         
    textfont={chineseBoldHeiti}, 
    skip=0.5em,       
}
%% 正文前罗马字体编号
\newcommand{\frontmatter}{
    \pagenumbering{Roman}
} 
\newcommand{\mainmatter} {%
    \clearpage{\cleardoublepage}
    \pagenumbering{arabic}
}

%%% 配置主目录中显示的节。
% 节
\titleformat{\section}{\sanhao\hei}{\thesection}{1em}{}
\titlespacing{\section}{0pt}{0pt}{12pt}
% 子节
\titleformat{\subsection}{\xiaosi\hei}{\thesubsection}{0.5em}{}
\titlespacing{\subsection}{0pt}{6pt}{6pt}
% 孙子节
\titleformat{\subsubsection}{\xiaosi\hei}{\thesubsubsection}{0.5em}{}
\titlespacing{\subsubsection}{0pt}{6pt}{6pt}

\renewcommand{\headrulewidth}{0pt}




%%% 制作前几页
%%% 定义特有字段以及特殊的变量和命令
\makeatletter
    %%% 封面配置
    % 论文类型
    \def\thesistype#1{\def\@thesistype{#1}}\def\@thesistype{}
    % 封面页标题栏的第一行
    \def\titleOnTheFirstLine#1{\def\@titleOnTheFirstLine{#1}}\def\@titleOnTheFirstLine{}
    % 封面页标题栏的第二行
    \def\titleOnTheSecondLine#1{\def\@titleOnTheSecondLine{#1}}\def\@titleOnTheSecondLine{}
    % 中文标题,用于摘要页
    % 封面页标题栏的第一行
    \def\titleOnTheFirstLine#1{\def\@titleOnTheFirstLine{#1}}\def\@titleOnTheFirstLine{}
    % 封面页标题栏的第二行
    \def\titleOnTheSecondLine#1{\def\@titleOnTheSecondLine{#1}}\def\@titleOnTheSecondLine{}
    % 中文标题,用于摘要页
    \def\title#1{\def\@title{#1}}\def\@title{}
    % 英文标题,用于英文摘要页
    % 英文标题,用于英文摘要页
    \def\etitle#1{\def\@etitle{#1}}\def\@etitle{}
    % 文章作者
    \def\author#1{\def\@author{#1}}\def\@author{}
    % 学号
    \def\studentid#1{\def\@studentid{#1}}\def\@studentid{}
    % 专业
    \def\subject#1{\def\@subject{#1}}\def\@subject{}
    % 院系
    \def\faculty#1{\def\@faculty{#1}}\def\@faculty{}
    % 指导老师
    \def\teacher#1{\def\@teacher{#1}}\def\@teacher{}
    % 文章日期
    \def\date#1{\def\@date{#1}}\def\@date{}
    %%% 页眉配置
    % 页眉说明
    \def\heading#1{\def\@heading{#1}}\def\@heading{}
    %%% 摘要配置
    % 中文摘要
    \long\def\cabstract#1{\long\def\@cabstract{#1}}\long\def\@cabstract{}
    % 英文摘要
    \long\def\eabstract#1{\long\def\@eabstract{#1}}\long\def\@eabstract{}
    %%% 关键词配置
    % 中文关键词
    \def\ckeywords#1{\def\@ckeywords{#1}}\def\@ckeywords{}
    % 英文关键词
    \def\ekeywords#1{\def\@ekeywords{#1}}\def\@ekeywords{}

    % 必须存在此句。
    \newlength{\@title@width}
    
    \newcommand{\makeheaders} {
        \setlength{\tabcolsep}{0em} % 做之前紧凑
        % 封面
        %% 目前检出有点问题
        \begin{titlepage}
            \centering
            % 湖南大学 毛主席题字
            \begin{figure*}[htp]
                \centering
                \includegraphics[width=0.5\textwidth]{figures/hnu}
            \end{figure*}
            % 高清校徽
            \vspace{-1em}
            \begin{overpic}{figures/HSVGPDF}
                \put(4,75){\yihao \textbf{\texttt{HUNAN UNIVERSITY}}}
                \put(-35,40){\song\chuhao{\@thesistype}}
            \end{overpic}\par
            \begin{table}[H]
                \centering
                \begin{tabular}{rrc}
                    % multicolumn 必须是第一条指令,前面只能是字体而不掺杂任何的命令。
                    \multicolumn{2}{r}{\xiaoer\hei{论文(设计)题目:}}\rule{0pt}{1.05cm} & \underline{\parbox[c]{6cm}{\xiaoer\hei{\@titleOnTheFirstLine}}} \\
                    &\multicolumn{1}{r}{}\rule{0pt}{1.05cm}      & \underline{\parbox[c]{6cm}{\xiaoer\hei{\@titleOnTheSecondLine}}} \\
                    \rule{0pt}{1.05cm}&\xiaosi\hei{学~生~姓~名 :} & \underline{\parbox[c]{6cm}{\centerline{\xiaosi\song{\@author}}}} \\
                    \rule{0pt}{1.05cm}&\xiaosi\hei{学~生~学~号 :} & \underline{\parbox[c]{6cm}{\centerline{\xiaosi\song{\@studentid}}}} \\
                    \rule{0pt}{1.05cm}&\xiaosi\hei{专~业~班~级 :} & \underline{\parbox[c]{6cm}{\centerline{\xiaosi\song{\@subject}}}} \\
                    \rule{0pt}{1.05cm}&\xiaosi\hei{学~院~名~称 :} & \underline{\parbox[c]{6cm}{\centerline{\xiaosi\song{\@faculty}}}} \\
                    \rule{0pt}{1.05cm}&\xiaosi\hei{指~导~老~师 :} & \underline{\parbox[c]{6cm}{\centerline{\xiaosi\song{\@teacher}}}}\\
                \end{tabular}
            \end{table}\par
            \vspace{2em}
            \centering{\xiaosi\hei{\the\year{} 年 \the\month{} 月 \quad 日}}
        \end{titlepage}
        \clearpage
    
        %% 从此往后页眉的格式基本固定。
        \def\headrule { 
            \if @fancyplain
                \let\headrulewidth\plainheadrulewidth
            \fi
            \hrule\@height 1.0pt \@width\headwidth\vskip1pt % 上面线为1pt粗
            \hrule\@height 0.5pt \@width\headwidth          % 下面0.5pt粗
            \vskip-2\headrulewidth\vskip-1pt                % 两条线的距离1pt
            \vspace{7mm}                                    % 双线与下面正文之间的垂直间距
        }
        \pagestyle{fancy}
        \fancyhf{}                      % 清空所有fancy样式,方便使页眉左部和右部置空
        \chead{\song\xiaowu{\@heading}} % 页眉上的字体大小
        \cfoot{\song\wuhao{\thepage}}   % 页码大小

        %% 插入目录之中
        % 授权书
        \addcontentsline{toc}{section}{毕业论文(设计)原创性声明和毕业论文(设计)版权使用授权书} { 
            \centerline{\hei\xiaoer{湖\hspace{0.5em}南\hspace{0.5em}大\hspace{0.5em}学}}\par
            \begin{center} \hei\xiaoer{毕业论文(设计)原创性声明} \end{center}\par
            \song\defaultfont{
                本人郑重声明:所呈交的论文(设计)是本人在导师的指导下独立进行研究所取得的研究成果。
                除了文中特别加以标注引用的内容外,本论文(设计)不包含任何其他个人或集体已经发表或撰写的成果作品。
                对本文的研究做出重要贡献的个人和集体,均已在文中以明确方式标明。
                本人完全意识到本声明的法律后果由本人承担。
            }\par\vspace{10pt}
            \song\xiaosi{学生签名:\hspace{10.5em}日期: \the\year{} 年 \hspace{.5em} 月 \hspace{.5em} 日}
            \vspace{0.5cm}
            \begin{center}\hei\xiaoer{毕业论文(设计)版权使用授权书}\end{center}
            \par\vspace{10pt}
            \song\defaultfont{
                本毕业论文(设计)作者完全了解学校有关保留、使用论文(设计)的规定,
                同意学校保留并向国家有关部门或机构送交论文(设计)的复印件和电子版,
                允许论文(设计)被查阅和借阅。本人授权湖南大学可以将本论文(设计)的全部或部分内容编入有关数据库进行检索,
                可以采用影印、缩印或扫描等复制手段保存和汇编本论文(设计)。
            }\par
            \song\defaultfont{{本论文(设计)属于}}\par
            \vspace{-0.25em}\hspace{7.5em}1、保 密□,在\underline{\hspace{4em}}年解密后适用于本授权书。\par
            \vspace{-0.25em}\hspace{7.5em}2、不保密□。\par
            \vspace{-0.25em}\hspace{7.5em}(请在以上相应方框内打``$\surd$'')\par
            \vspace{1em}
            \song\xiaosi{学生签名:\hspace{10.5em}日期: \the\year{} 年 \hspace{.5em} 月 \hspace{.5em} 日}
            \par\vspace{0.25em}
            \song\xiaosi{导师签名:\hspace{10.5em}日期: \the\year{} 年 \hspace{.5em} 月 \hspace{.5em} 日}
        }
        \clearpage

        % 中文摘要  
        \addcontentsline{toc}{section}{摘~~要} {
            \begin{center}
                \hei\xiaoer{\@title}
            \end{center}
            \vspace{-10pt}
            \begin{center}
                \hei\xiaoer\ 摘\qquad 要
            \end{center}\par
            \song\defaultfont\@cabstract
            \vspace{\baselineskip}
            \hangafter=1\hangindent=52.3pt
            \newline\noindent
            {\hei\sihao{关键词:} \hei\xiaosi\@ckeywords}
        }
        \clearpage
        
        % 英文摘要
        \addcontentsline{toc}{section}{Abstract} {
            \begin{center}
                \xiaoer{\textbf{\@etitle}}
            \end{center}\par
            \vspace{-12pt}
            \begin{center}
                \xiaoer\textbf{Abstract}
            \end{center}\par
            \vspace{-6pt}
            \defaultfont{\@eabstract}
            \vspace{\baselineskip}
            \hangafter=1\hangindent=60pt
            \newline\noindent
            {\xiaosi\textbf{Key Words:} \@ekeywords}
        }
        \newpage
        % 图表以及目录
        \setlength{\tabcolsep}{0.5em} % 做完上述部分后,表格默认每个格子 0.5em 宽
    }

    \fancyhf{} % 清空所有fancy样式,方便使页眉左部和右部置空
    \pagestyle{fancy}
    \fancypagestyle{plain}{
        \fancyhead[C]{\song\xiaowu \@heading}
        \fancyfoot[C]{\song\xiaowu \thepage}
    }
    
\makeatother
