% !Mode:: "TeX:UTF-8"
%%% 定义字体和字号
\newcommand{\song}{\CJKfamily{song}}    % 宋体
\newcommand{\hei}{\CJKfamily{hei}}      % 黑体
\newcommand{\fs}{\CJKfamily{fs}}        % 仿宋体
\newcommand{\kai}{\CJKfamily{kai}}      % 楷体
\newcommand{\li}{\CJKfamily{li}}        % 隶书
%%%% 字号换算规范
%%%% 中文常用字号来代替国际用点数/磅/英寸/pt。
%% 参考 https://www.runoob.com/w3cnote/px-pt-em-convert-table.html
% 初号      42pt
% 小初号    36pt
% 一号      28pt
% 小一
% 二号      21pt
% 小二号    18pt
% 三号      15.75pt
% 四号      14pt
% 小四号    12pt
% 五号      10.5pt
% 小五号    9pt
% 六号      7.875pt
% 七号      5.25pt
\newcommand{\biaoti}{\fontsize{45pt}{45pt}\selectfont}      % 一号, 1.倍行距
\newcommand{\yihao}{\fontsize{28pt}{28pt}\selectfont}       % 一号, 1.倍行距
\newcommand{\xiaoyi}{\fontsize{24pt}{24pt}\selectfont}      % 小一, 1.倍行距
\newcommand{\erhao}{\fontsize{22pt}{22pt}\selectfont}       % 二号, 1.倍行距
\newcommand{\xiaoer}{\fontsize{17pt}{17pt}\selectfont}      % 小二, 单倍行距
\newcommand{\sanhao}{\fontsize{16pt}{16pt}\selectfont}      % 三号, 1.倍行距
\newcommand{\xiaosan}{\fontsize{15pt}{15pt}\selectfont}     % 小三, 1.倍行距
\newcommand{\sihao}{\fontsize{14pt}{14pt}\selectfont}       % 四号, 1.0倍行距
\newcommand{\xiaosi}{\fontsize{12.5pt}{12.5pt}\selectfont}  % 小四, 1.倍行距
\newcommand{\wuhao}{\fontsize{10.5pt}{10.5pt}\selectfont}   % 五号, 单倍行距
\newcommand{\xiaowu}{\fontsize{9pt}{9pt}\selectfont}        % 小五, 单倍行距
\setlength{\headheight}{20pt}
\CJKtilde

\setitemize{leftmargin=3em,itemsep=0em,partopsep=0em,parsep=0em,topsep=-0em}
\setenumerate{leftmargin=3em,itemsep=0em,partopsep=0em,parsep=0em,topsep=0em}


\makeatletter
    \renewcommand\normalsize{
        \@setfontsize\normalsize{12pt}{12pt}
        \setlength\abovedisplayskip{4pt}
        \setlength\abovedisplayshortskip{4pt}
        \setlength\belowdisplayskip{\abovedisplayskip}
        \setlength\belowdisplayshortskip{\abovedisplayshortskip}
        \let\@listi\@listI}
    \def\defaultfont{\renewcommand{\baselinestretch}{1.65}\normalsize\selectfont}
    \setlength{\baselineskip}{20pt}
    \renewcommand{\CJKglue}{\hskip 0.5pt plus \baselineskip}
\makeatother

\def\functor{\relax}
\let\functor\addcontentsline
\gdef\addcontentsline{\phantomsection\functor} % 这样就能修改addcontentsline的功能,使其额外插入。

%%% 配置目录
\renewcommand*{\contentsname}{\hfill 目\qquad 录 \hfill}
\setcounter{secnumdepth}{3} % 编号深度 3
\setcounter{tocdepth}{3}   % 目录项深度为 3

% \titlecontents{chapter}[0em]{\xiaosi\hei}%
% {\thecontentslabel\quad}{} %
% {\titlerule*[5pt]{$\cdot$}\xiaosi\contentspage}

% 让 section 从 1 开始计数。
\renewcommand\thesection{\arabic{section}}
\titlecontents{section}[]{\xiaosi\hei} %
{\thecontentslabel\ \ }{} % \hspace{.25em}
{\titlerule*[5pt]{$\cdot$}\xiaosi\contentspage}

\titlecontents{subsection}[2em]{\xiaosi\song} %
{\thecontentslabel\ }{} %
{\hspace{.25em}\titlerule*[5pt]{$\cdot$}\xiaosi\contentspage}

\titlecontents{subsubsection}[4em]{\xiaosi\song} %
{\thecontentslabel\ }{} %
{\hspace{.25em}\titlerule*[5pt]{$\cdot$}\xiaosi\contentspage}

\renewcommand{\cftdotsep}{1.1}
\renewcommand*{\listfigurename}{\hfill 插图索引 \hfill}
\setcounter{lofdepth}{1}
\renewcommand*{\listtablename}{\hfill 附表索引 \hfill}
\setcounter{secnumdepth}{4}
\setlength{\parindent}{2em}

\renewcommand{\figurename}{图}
\renewcommand{\tablename}{表}

%% 正文前罗马字体编号
\newcommand{\frontmatter}{
    % \clearpage{\cleardoublepage}
    \pagenumbering{Roman}
} 
\newcommand\mainmatter{%
    \clearpage{\cleardoublepage}
    \pagenumbering{arabic}
}

%%% 配置节
% 节
\titleformat{\section}{\sanhao\hei}{\thesection}{1em}{}
\titlespacing{\section}{0pt}{0pt}{12pt}
% 子节
\titleformat{\subsection}{\xiaosi\hei}{\thesubsection}{0.5em}{}
\titlespacing{\subsection}{0pt}{6pt}{6pt}
% 孙子节
\titleformat{\subsubsection}{\xiaosi\hei}{\thesubsubsection}{0.5em}{}
\titlespacing{\subsubsection}{0pt}{6pt}{6pt}


%%% 制作前几页
\renewcommand{\headrulewidth}{0pt}


%%% 定义特有字段以及特殊的变量和命令
\makeatletter
%% 封面配置
    % 论文类型
    \def\thesistype#1{\def\@thesistype{#1}}\def\@thesistype{}
    % 中文标题
    \def\title#1{\def\@title{#1}}\def\@title{}
    % 英文标题
    \def\etitle#1{\def\@etitle{#1}}\def\@etitle{}
    % 文章作者
    \def\author#1{\def\@author{#1}}\def\@author{}
    % 学号
    \def\studentid#1{\def\@studentid{#1}}\def\@studentid{}
    % 专业
    \def\subject#1{\def\@subject{#1}}\def\@subject{}
    % 院系
    \def\faculty#1{\def\@faculty{#1}}\def\@faculty{}
    % 指导老师
    \def\teacher#1{\def\@teacher{#1}}\def\@teacher{}
    % 文章日期
    \def\date#1{\def\@date{#1}}\def\@date{}

    %%% 配置授权与声明页
    % 声明标题
    \def\declaredtitle#1{\def\@declaredtitle{#1}}\def\@declaredtitle{}
    % 声明内容
    \def\declaration#1{\def\@declaration{#1}}\def\@declaration{}
    % 授权书名称
    \def\authorizedtitle#1{\def\@authorizedtitle{#1}}\def\@authorizedtitle{}
    % 授权内容
    \def\authorization#1{\def\@authorization{#1}}\def\@authorization{}
    %% 配置签名
    \def\studentsign#1{\def\@studentsign{#1}}\def\@studentsign{}
    \def\teachersign#1{\def\@teachersign{#1}}\def\@teachersign{}
    % 即日期
    \def\cdatename#1{\def\@cdatename{#1}}\def\@cdatename{}

    %% 页眉配置
    % 页眉说明
    \def\heading#1{\def\@heading{#1}}\def\@heading{}

    %% 摘要配置
    % 中文摘要
    \long\def\cabstract#1{\long\def\@cabstract{#1}}\long\def\@cabstract{}
    % 英文摘要
    \long\def\eabstract#1{\long\def\@eabstract{#1}}\long\def\@eabstract{}
    
    %% 关键词配置
    % 中文关键词
    \def\ckeywords#1{\def\@ckeywords{#1}}\def\@ckeywords{}
    % 英文关键词
    \def\ekeywords#1{\def\@ekeywords{#1}}\def\@ekeywords{}

    % 必须存在此句。
    \newlength{\@title@width}
    
    \newcommand{\makeheaders}{
        %% 写完以后测试一下这一句的效果 \phantomsection
        % 封面
        % \addcontentsline{toc}{chpater}{\@title} 
        \begin{titlepage}
            \centering
            % 湖南大学 毛主席题字
            \begin{figure}[h]
                \centering
                \includegraphics[width=0.5\textwidth]{figures/hnu}
            \end{figure}
            % 高清校徽
            \begin{overpic}{figures/HSVGPDF}
                \put(5,75){\li\yihao \textbf{\texttt{HUNAN UNIVERSITY}}}
                \put(-35,40){\song\biaoti \@thesistype}
            \end{overpic}
            \vspace{1cm}
            \vspace{\baselineskip}
            \setlength{\@title@width}{6.5cm}
            \begin{spacing}{2.1}
                % 这里需要自己来分。
                \put(47,60){\xiaoer\hei{论文(设计)题目:} \xiaoer\hei\underline{\makebox[\@title@width][l]{WebAssembly和eBPF}}} \\
                \put(98,80){\xiaoer\hei{\qquad\qquad } \xiaoer\hei\underline{\makebox[\@title@width][l]{的安全机制优化研究}}} \\
                
                \xiaosi\hei{学~生~姓~名:} \xiaosi\song\underline{\makebox[\@title@width][c]{\@author}} \\
                \xiaosi\hei{学~生~学~号:} \xiaosi\song\underline{\makebox[\@title@width][c]{\@studentid}} \\
                \xiaosi\hei{专~业~班~级:} \xiaosi\song\underline{\makebox[\@title@width][c]{\@subject}} \\
                \xiaosi\hei{学~院~名~称:} \xiaosi\song\underline{\makebox[\@title@width][c]{\@faculty}} \\
                \xiaosi\hei{指~导~老~师:} \xiaosi\song\underline{\makebox[\@title@width][c]{\@teacher}}\\
                \xiaoer\hei{\qquad\qquad\quad}&\\ 
                \xiaosi\hei{\makebox[\@title@width][l]{\qquad \qquad \qquad 2025 年 6 月}} \\
            \end{spacing}
            \end{titlepage}
        
        \clearpage
    
        %% 从此往后页眉的格式基本固定。
        \def\headrule { 
            
            \if@fancyplain
                \let\headrulewidth\plainheadrulewidth
            \fi
            \hrule\@height 1.0pt \@width\headwidth\vskip1pt %上面线为1pt粗
            \hrule\@height 0.5pt\@width\headwidth  	%下面0.5pt粗
            \vskip-2\headrulewidth\vskip-1pt 		%两条线的距离1pt
                
            \vspace{7mm} %双线与下面正文之间的垂直间距
        }     
        \pagestyle{fancy}
        \fancyhf{} % 清空所有fancy样式,方便使页眉左部和右部置空
        \chead{\song\xiaowu \@heading}
        \cfoot{\song\xiaowu \thepage}

        %% 插入目录之中
        % 授权书
        \addcontentsline{toc}{section}{\@declaredtitle和\@authorizedtitle} { 
            \vspace{0.5\baselineskip}
            \begin{center}
                \hei\xiaoer{湖南大学}
            \end{center}\par

            \vspace{0.5\baselineskip}
            \begin{center}
                \hei\xiaoer{\@declaredtitle}
            \end{center}\par

            \vspace{0.5\baselineskip}
            \song\defaultfont{\@declaration}\par
            \vspace{0.4\baselineskip}
            
            \song\xiaosi
            \@studentsign \makebox[3cm][s]{} \qquad\qquad\qquad
            \@cdatename 20\makebox[0.5cm][s]{} 年 \makebox[0.5cm][s]{} 月 \makebox[0.5cm][s]{} 日
            
            \vspace*{1cm}
            \begin{center}\hei\xiaoer{\@authorizedtitle}\end{center}\par
            \vspace{1.2\baselineskip}
            \song\defaultfont{\@authorization}\par
            \song\defaultfont{本论文(设计)属于}\par
            % 用表格来放置保密签订协议。
            \begin{tabular}{lll}
                \qquad\qquad& \qquad\quad\qquad&\ \ \ \ 1、保 密□,在\underline{\qquad\qquad}年解密后适用于本授权书。 \\
                \qquad\qquad& \qquad\quad\qquad&\ \ \ \ 2、不保密□。                                       \\
                \qquad\qquad& \qquad\quad\qquad&\ \ \ \ (请在以上相应方框内打``$\surd$'')
            \end{tabular}
            \par
            \vspace{2\baselineskip}
            \song\xiaosi
            \@studentsign \makebox[3cm][s]{} \qquad\qquad\qquad  
            \@cdatename 20\makebox[0.5cm][s]{} 年 \makebox[0.5cm][s]{} 月 \makebox[0.5cm][s]{} 日 \\
            \indent
            \@teachersign \makebox[3cm][s]{} \qquad\qquad\qquad  
            \@cdatename 20\makebox[0.5cm][s]{} 年 \makebox[0.5cm][s]{} 月 \makebox[0.5cm][s]{} 日
        }
        \clearpage

        % 中文摘要  
        \addcontentsline{toc}{section}{摘~~要} {
            \begin{center}
                \hei\xiaoer{\@title}
            \end{center}\par
            \vspace{0.5\baselineskip}
            \begin{center}
                \hei\xiaoer\ 摘\qquad 要
            \end{center}\par
            \vspace{0.5\baselineskip}
            \song\defaultfont\@cabstract
            \vspace{\baselineskip}
            \hangafter=1\hangindent=52.3pt\noindent
            \newline\noindent
            {\hei\sihao{关键词:} \hei\xiaosi\@ckeywords}
        }
        \clearpage
        
        % 英文摘要
        \addcontentsline{toc}{section}{Abstract} {
            \begin{center}
                \xiaoer{\textbf{\@etitle}}
            \end{center}\par
            \vspace{0.5\baselineskip}
            \begin{center}
                \xiaoer\textbf{Abstract}
            \end{center}\par
            % \vspace{\baselineskip}
            \@eabstract
            \vspace{\baselineskip}
            \hangafter=1\hangindent=60pt\noindent
            \newline\noindent
            {\xiaosi\textbf{Key Words:} \@ekeywords}
        }
        \clearpage
        % 图表以及目录
    }

    \fancyhf{} % 清空所有fancy样式,方便使页眉左部和右部置空
    \pagestyle{fancy}
    \fancypagestyle{plain}{
        \fancyhead[C]{\song\xiaowu \@heading}
        \fancyfoot[C]{\song\xiaowu \thepage}
    }
\makeatother

% \renewcommand{\bibname}{\centering\hei\sanhao 参考文献}
\setlength{\bibsep}{0em} % 参考文献项目间距
\setlength{\bibhang}{1em} % 
\bibpunct{[}{]}{,}{s}{}{,} % 角标设置




