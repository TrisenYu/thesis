%%% 页边距定义
% 上边距 3cm
% 下边距 2.5cm
% 页眉到上边缘的距离 1.5cm
% 页脚到下边缘的距离 1.75cm
\setlength{\voffset}{0cm}
\setlength{\topmargin}{1.5cm-1in}
\setlength{\headheight}{13pt} % 页眉可以放得大一点
\setlength{\headsep}{1.5cm-12pt}
\setlength{\textheight}{\paperheight-5.5cm}
\setlength{\footskip}{0.75cm}

%%% 控制行距
% \setlength{\parindent}{4em} % 段落缩进
% \setlength{\parskip}{1.5em} % 段落间距
\renewcommand{\baselinestretch}{1.5}

%%% 定义字体
\newcommand{\song}{\CJKfamily{song}}    % 宋体 rm
\newcommand{\hei}{\CJKfamily{hei}}      % 黑体 bf 
\newcommand{\fs}{\CJKfamily{fs}}        % 仿宋体 tt
\newcommand{\kai}{\CJKfamily{kai}}      % 楷体 it
\newcommand{\li}{\CJKfamily{li}}        % 隶书


%%%% 字号换算规范
%%%% 中文常用字号来代替国际用点数/磅/英寸/pt。
%% 参考 https://www.runoob.com/w3cnote/px-pt-em-convert-table.html
\newcommand{\chuhao}{\fontsize{42pt}{42pt}\selectfont}      % 初号, 1.倍行距
\newcommand{\yihao}{\fontsize{26pt}{26pt}\selectfont}       % 一号, 1.倍行距
\newcommand{\xiaoyi}{\fontsize{24pt}{24pt}\selectfont}      % 小一, 1.倍行距
\newcommand{\erhao}{\fontsize{22pt}{22pt}\selectfont}       % 二号, 1.倍行距
\newcommand{\xiaoer}{\fontsize{18pt}{18pt}\selectfont}      % 小二, 单倍行距
\newcommand{\sanhao}{\fontsize{16pt}{16pt}\selectfont}      % 三号, 1.倍行距
\newcommand{\xiaosan}{\fontsize{15pt}{15pt}\selectfont}     % 小三, 1.倍行距
\newcommand{\sihao}{\fontsize{14pt}{14pt}\selectfont}       % 四号, 1.0倍行距
\newcommand{\xiaosi}{\fontsize{12pt}{12pt}\selectfont}      % 小四, 1.倍行距
\newcommand{\wuhao}{\fontsize{11pt}{11pt}\selectfont}       % 五号, 单倍行距
\newcommand{\xiaowu}{\fontsize{9pt}{9pt}\selectfont}        % 小五, 单倍行距
% \setlength{\headheight}{20pt} 
\CJKtilde

\setitemize{leftmargin=3em,itemsep=0em,partopsep=0em,parsep=0em,topsep=-0em}
\setenumerate{leftmargin=3em,itemsep=0em,partopsep=0em,parsep=0em,topsep=0em}
\captionsetup[table] {
    labelfont={bf, small, stretch=1.25}
}
\captionsetup[figure] {
    labelfont={bf, small, stretch=1.25}
}


\makeatletter
    \renewcommand\normalsize{
        \@setfontsize\normalsize{12pt}{12pt}
        \setlength\abovedisplayskip{4pt}
        \setlength\abovedisplayshortskip{4pt}
        \setlength\belowdisplayskip{\abovedisplayskip}
        \setlength\belowdisplayshortskip{\abovedisplayshortskip}
        \let\@listi\@listI}
    \def\defaultfont{\renewcommand{\baselinestretch}{2}\normalsize\selectfont}
    \setlength{\baselineskip}{20pt}
    \renewcommand{\CJKglue}{\hskip 0.5pt plus \baselineskip}
\makeatother

\def\functor{\relax}
\let\functor\addcontentsline
\gdef\addcontentsline{\phantomsection\functor} % 这样就能修改addcontentsline的功能,使其额外插入。

%%% 配置目录
\renewcommand*{\contentsname}{\hfill 目\qquad\quad录 \hfill}
\setcounter{secnumdepth}{3} % 编号深度 3
\setcounter{tocdepth}{3}    % 目录项深度为 3

% 让 section 从 1 开始计数。
\renewcommand{\cftdot}{$...$} % 点的样式。
\renewcommand{\cftdotsep}{1.5}
\renewcommand\thesection{\arabic{section}}
\titlecontents{section}[0em] % 标题级别 标题位置 (左间距,理解为缩进)
{\xiaosi\hei\vspace{0.5em}}  % % 标题样式
{\textbf{\thecontentslabel}\hspace{0.5em}} % 标题格式 (设置标题的整体格式,如字体尺寸、粗细、与上一个标题的垂直距离等,可空置)
{} 
{\titlerule*[10pt]{\cftdot}\xiaosi\contentspage} % 指引线与页码 (设置标题与页码之间的指引线样式以及页码的格式,该参数如果空置,标题将无指引线和页码)

\titlecontents{subsection}[2em]{\xiaosi\song\vspace{0.5em}} %
{\thecontentslabel\hspace{0.5em}}{} %
{\titlerule*[10pt]{\cftdot}\xiaosi\contentspage}

\titlecontents{subsubsection}[4em]{\xiaosi\song\vspace{0.5em}} %
{\thecontentslabel\hspace{0.5em}}{} %
{\titlerule*[10pt]{\cftdot}\xiaosi\contentspage}

\renewcommand{\cftdotsep}{1.1}
\renewcommand*{\listfigurename}{\hfill 插图索引 \hfill}
\setcounter{lofdepth}{1}
\renewcommand*{\listtablename}{\hfill 附表索引 \hfill}
\setcounter{secnumdepth}{4}
\setlength{\parindent}{2em}

\renewcommand{\cftfigleader}{\cftdotfill{\cftdotsep}}
\renewcommand{\cfttableader}{\cftdotfill{\cftdotsep}}


\renewcommand{\figurename}{\textbf{\wuhao\songti{图}}}
\renewcommand{\tablename}{\textbf{\wuhao\heiti{表}}}

%% 正文前罗马字体编号
\newcommand{\frontmatter}{
    % \clearpage{\cleardoublepage}
    \pagenumbering{Roman}
} 
\newcommand{\mainmatter} {%
    \clearpage{\cleardoublepage}
    \pagenumbering{arabic}
}

%%% 配置节
% 节
\titleformat{\section}{\sanhao\hei}{\thesection}{1em}{}
\titlespacing{\section}{0pt}{0pt}{12pt}
% 子节
\titleformat{\subsection}{\xiaosi\hei}{\thesubsection}{0.5em}{}
\titlespacing{\subsection}{0pt}{6pt}{6pt}
% 孙子节
\titleformat{\subsubsection}{\xiaosi\hei}{\thesubsubsection}{0.5em}{}
\titlespacing{\subsubsection}{0pt}{6pt}{6pt}

\renewcommand{\headrulewidth}{0pt}

%%% 制作前几页
%%% 定义特有字段以及特殊的变量和命令
\makeatletter
    %%% 封面配置
    % 论文类型
    \def\thesistype#1{\def\@thesistype{#1}}\def\@thesistype{}
    % 封面页标题栏的第一行
    \def\titleOnTheFirstLine#1{\def\@titleOnTheFirstLine{#1}}\def\@titleOnTheFirstLine{}
    % 封面页标题栏的第二行
    \def\titleOnTheSecondLine#1{\def\@titleOnTheSecondLine{#1}}\def\@titleOnTheSecondLine{}
    % 中文标题,用于摘要页
    % 封面页标题栏的第一行
    \def\titleOnTheFirstLine#1{\def\@titleOnTheFirstLine{#1}}\def\@titleOnTheFirstLine{}
    % 封面页标题栏的第二行
    \def\titleOnTheSecondLine#1{\def\@titleOnTheSecondLine{#1}}\def\@titleOnTheSecondLine{}
    % 中文标题,用于摘要页
    \def\title#1{\def\@title{#1}}\def\@title{}
    % 英文标题,用于英文摘要页
    % 英文标题,用于英文摘要页
    \def\etitle#1{\def\@etitle{#1}}\def\@etitle{}
    % 文章作者
    \def\author#1{\def\@author{#1}}\def\@author{}
    % 学号
    \def\studentid#1{\def\@studentid{#1}}\def\@studentid{}
    % 专业
    \def\subject#1{\def\@subject{#1}}\def\@subject{}
    % 院系
    \def\faculty#1{\def\@faculty{#1}}\def\@faculty{}
    % 指导老师
    \def\teacher#1{\def\@teacher{#1}}\def\@teacher{}
    % 文章日期
    \def\date#1{\def\@date{#1}}\def\@date{}
    %%% 配置授权与声明页
    % 声明标题
    \def\declaredtitle#1{\def\@declaredtitle{#1}}\def\@declaredtitle{}
    % 声明内容
    \def\declaration#1{\def\@declaration{#1}}\def\@declaration{}
    % 授权书名称
    \def\authorizedtitle#1{\def\@authorizedtitle{#1}}\def\@authorizedtitle{}
    % 授权内容
    \def\authorization#1{\def\@authorization{#1}}\def\@authorization{}
    %% 配置签名
    \def\studentsign#1{\def\@studentsign{#1}}\def\@studentsign{}
    \def\teachersign#1{\def\@teachersign{#1}}\def\@teachersign{}
    % 即日期
    \def\cdatename#1{\def\@cdatename{#1}}\def\@cdatename{}
    %%% 页眉配置
    % 页眉说明
    \def\heading#1{\def\@heading{#1}}\def\@heading{}
    %%% 摘要配置
    % 中文摘要
    \long\def\cabstract#1{\long\def\@cabstract{#1}}\long\def\@cabstract{}
    % 英文摘要
    \long\def\eabstract#1{\long\def\@eabstract{#1}}\long\def\@eabstract{}
    %%% 关键词配置
    % 中文关键词
    \def\ckeywords#1{\def\@ckeywords{#1}}\def\@ckeywords{}
    % 英文关键词
    \def\ekeywords#1{\def\@ekeywords{#1}}\def\@ekeywords{}

    % 必须存在此句。
    \newlength{\@title@width}
    
    \newcommand{\makeheaders}{
        % 封面
        % \addcontentsline{toc}{chpater}{\@title} 
        \begin{titlepage}
            \centering
            % 湖南大学 毛主席题字
            \begin{figure}[h]
                \centering
                \includegraphics[width=0.5\textwidth]{figures/hnu}
            \end{figure}
            % 高清校徽
            \begin{overpic}{figures/HSVGPDF}
                \put(5,75){\yihao \textbf{\texttt{HUNAN UNIVERSITY}}}
                \put(-35,40){\song\chuhao{\@thesistype}}
            \end{overpic}
            \vspace{1cm}
            \vspace{\baselineskip}
            \setlength{\@title@width}{6.5cm}
            \begin{spacing}{2.1}
                \put(37,60){\xiaoer\hei{论文(设计)题目:} \underline{\makebox[\@title@width][l]{\xiaoer\hei{\@titleOnTheFirstLine}}}} \par
                \put(91,80){\xiaoer\hei{\qquad\qquad } \underline{\makebox[\@title@width][l]{\xiaoer\hei{\@titleOnTheSecondLine}}}} \par
                \xiaosi\hei{学~生~姓~名 :} \xiaosi\song\underline{\makebox[\@title@width][c]{\@author}} \par
                \xiaosi\hei{学~生~学~号 :} \xiaosi\song\underline{\makebox[\@title@width][c]{\@studentid}} \par
                \xiaosi\hei{专~业~班~级 :} \xiaosi\song\underline{\makebox[\@title@width][c]{\@subject}} \par
                \xiaosi\hei{学~院~名~称 :} \xiaosi\song\underline{\makebox[\@title@width][c]{\@faculty}} \par
                \xiaosi\hei{指~导~老~师 :} \xiaosi\song\underline{\makebox[\@title@width][c]{\@teacher}}\par
                \xiaoer\hei{\qquad\qquad\quad} \par
                \xiaosi\hei{\makebox[\@title@width][l]{\qquad \qquad \qquad \the\year{} 年 \the\month{} 月}}
            \end{spacing}
        \end{titlepage}
        
        \clearpage
    
        %% 从此往后页眉的格式基本固定。
        \def\headrule { 
            \if @fancyplain
                \let\headrulewidth\plainheadrulewidth
            \fi
            \hrule\@height 1.0pt \@width\headwidth\vskip1pt %上面线为1pt粗
            \hrule\@height 0.5pt \@width\headwidth          %下面0.5pt粗
            \vskip-2\headrulewidth\vskip-1pt 		        %两条线的距离1pt
                
            \vspace{7mm} %双线与下面正文之间的垂直间距
        }     
        \pagestyle{fancy}
        \fancyhf{} % 清空所有fancy样式,方便使页眉左部和右部置空
        \chead{\song\xiaowu{\@heading}} % 页眉上的字体大小
        \cfoot{\song\wuhao{\thepage}} % 页码大小

        %% 插入目录之中
        % 授权书
        \addcontentsline{toc}{section}{\@declaredtitle{和}\@authorizedtitle} { 
            \vspace{0.5\baselineskip}
            \begin{center}
                \hei\xiaoer{湖\hspace{0.5em}南\hspace{0.5em}大\hspace{0.5em}学}
            \end{center}\par
            \begin{center}
                \hei\xiaoer{\@declaredtitle}
            \end{center}\par
            \song\defaultfont{\@declaration}\par
            \vspace{0.5\baselineskip}

            \song\xiaosi
            \@studentsign \makebox[3cm][s]{} \qquad\qquad\qquad
            \@cdatename \the\year{} 年 \makebox[0.5cm][s]{} 月 \makebox[0.5cm][s]{} 日
            
            \vspace{0.5cm}
            \begin{center}\hei\xiaoer{\@authorizedtitle}\end{center}\par
            \vspace{0.5\baselineskip}
            \song\defaultfont{\@authorization}\par
            \song\defaultfont{{本论文(设计)属于}}\par
            % 用表格来放置保密签订协议。
            \begin{tabular}{lll}
                \qquad\qquad&\qquad\quad\qquad&\hspace{0.5em} 1、保 密□,在\underline{\qquad\qquad}年解密后适用于本授权书。 \\
                \qquad\qquad&\qquad\quad\qquad&\hspace{0.5em} 2、不保密□。                                       \\
                \qquad\qquad&\qquad\quad\qquad&\hspace{0.5em} (请在以上相应方框内打``$\surd$'')
            \end{tabular}
            \par
            \vspace{1\baselineskip}
            \song\xiaosi
            \@studentsign \makebox[3cm][s]{} \qquad\qquad\qquad  
            \@cdatename \the\year{} 年 \makebox[0.5cm][s]{} 月 \makebox[0.5cm][s]{} 日\par
            \vspace{0.25\baselineskip}
            \@teachersign \makebox[3cm][s]{} \qquad\qquad\qquad  
            \@cdatename \the\year{} 年 \makebox[0.5cm][s]{} 月 \makebox[0.5cm][s]{} 日
        }
        \clearpage

        % 中文摘要  
        \addcontentsline{toc}{section}{摘~~要} {
            \begin{center}
                \hei\xiaoer{\@title}
            \end{center}\par
            \vspace{0.5\baselineskip}
            \begin{center}
                \hei\xiaoer\ 摘\qquad 要
            \end{center}\par
            \vspace{0.5\baselineskip}
            \song\defaultfont\@cabstract
            \vspace{\baselineskip}
            \hangafter=1\hangindent=52.3pt
            \newline\noindent
            {\hei\sihao{关键词:} \hei\xiaosi\@ckeywords}
        }
        \clearpage
        
        % 英文摘要
        \addcontentsline{toc}{section}{Abstract} {
            \begin{center}
                \xiaoer{\textbf{\@etitle}}
            \end{center}\par
            \vspace{0.5\baselineskip}
            \begin{center}
                \xiaoer\textbf{Abstract}
            \end{center}\par
            \@eabstract
            \vspace{\baselineskip}
            \hangafter=1\hangindent=60pt
            \newline\noindent
            {\xiaosi\textbf{Key Words:} \@ekeywords}
        }
        \clearpage
        % 图表以及目录
    }

    \fancyhf{} % 清空所有fancy样式,方便使页眉左部和右部置空
    \pagestyle{fancy}
    \fancypagestyle{plain}{
        \fancyhead[C]{\song\xiaowu \@heading}
        \fancyfoot[C]{\song\xiaowu \thepage}
    }
\makeatother
