\section{相关工作}
\label{section:relatedWork}
\subsection{eBPF}
    扩展的伯克利包过滤器(extended Berkeley Packet Filter,eBPF)是一项基于事件驱动的Linux内核拓展技术,
    支持动态地将用户编写的代码通过eBPF虚拟机加载到内核态上执行,而无需修改内核代码或者插入内核模块
    \cite{sunFindingCorrectnessBugs2024, YIHeCrossContainer, riceLearningEBPFProgramming2023}。
    这一直接运行于内核态而无需通过切换用户态和内核态之间的能力,既减少了数据复制次数提高了内核执行性能,
    又充分扩展了内核的功能\cite{YIHeCrossContainer, ZhangZiJunLinuxXiTonge};
    eBPF的这种灵活性被用于多种特定的任务\cite{HaoValidating},
    如网络包过滤\cite{10.1145/3371038, TCPdump}、网络流量监控\cite{9110434}以及其它方面\cite{280870, 258973}。

    \subsubsection{eBPF运行流程}
        用户首先需要通过相应的eBPF框架来编写程序,通过\cite{riceLearningEBPFProgramming2023}
        eBPF程序运行时,只能操控11个寄存器(为R0至R10),其功能分别为:
        \begin{itemize}
            \item[(1)] R0:保存函数返回值和eBPF程序退出值;
            \item[(2)] R1$\sim$ R5:保存函数调用参数;
            \item[(3)] R6$\sim$ R9:保存调用函数的上下文;
            \item[(4)] R10: 只读的栈帧寄存器,用于访问栈。
        \end{itemize}

        同时,eBPF程序能在一个固定大小的栈上执行四种基本运算:加载、存储、算术运算和分支判断
        \cite{HaoValidating}。\nolinebreak
        运行BPF程序前,会经过编译工具链(如clang)编译为BPF字节码,
        而后由BPF系统调用加载到内核态等待通过eBPF验证器的静态验证
        \cite{zhengBpftimeUserspaceEBPF2023}。\nolinebreak
        通过验证后,会经由BPF Just-In-Time(JIT)编译器编译为可执行的机器码并被加载到虚拟机中执行。

    \subsubsection{eBPF的限制}
        由于eBPF程序运行在内核态,加载到内核中的代码在编写不当时可能会导致内核崩溃。
        为确保从用户态加载进内核的代码不会引起这种情况,eBPF会使用自身的验证器检查
        eBPF程序中的是否包含潜在的不安全操作和不可接受的性能开销。程序通过验证后,
        即时编译 (Just-In-Time Compilation, JIT)模块被用来将eBPF指令转化为主
        机指令以加快执行速度\cite{FuzzOnEBPF}。

        而仅eBPF的验证器的处理逻辑,就贡献了eBPF本身超过半数的CVE\cite{hive}。
        而且,当前的eBPF在性能上有所局限,eBPF的调用栈最大深度只允许为512,且经JIT
        编译后的指令数最多不能超过100万条;eBPF的验证器存在假阳性\cite{hive}和
        假阴性问题\cite{RJXB202312023}。假阳性即实现逻辑复杂但是不会影响其它程序
        正常运行的程序可能不会被验证通过,假阴性是指有的程序可能会
        % citation required.

\subsection{Web Assembly概况}
    Web Assembly(缩略为wasm)是一种为增强Web浏览器性能和拓展性而面向C,C++,rust
    和Golang等高级编程语言设计的中间编译目标(或称字节码)
    \cite{wasmCommunityGroup, lehmannWasabiFrameworkDynamically2019, 
        lehmannEverythingOldNew, bhansaliFirstLookCode2022, waseemIssuesTheirCauses2024}。
    因这种字节码并不能直接交由处理器译码执行,还需经过跨硬件指令集的转译,wasm有着跨平台的可扩展性
    \cite{lehmannEverythingOldNew, waseemIssuesTheirCauses2024, 
        lehmannWasabiFrameworkDynamically2019, JayProvablySafe, 
        WebAssemblySummaryOnSecurity, rayOverviewWebAssemblyIoT2023}。

    作为弥补JavaScript性能瓶颈的补充技术之一\cite{rayOverviewWebAssemblyIoT2023},
    wasm生成的字节码经转译后有接近原生机器码(即直接通过编译器编译得到的机器码)的执行性能
    \cite{haasBringingWebSpeed2017,johnsonWaVeVerifiablySecure2023}。
    即使如此,有研究\cite{JangdaNotsoFast}指出,由于这种转译增加了总体汇编指令的个数、
    访存加载数、分支数以及额外的安全检查(如栈溢出和间接执行检查),造成L1缓存不命中率的上升,
    wasm的性能相比原生编译的可执行程序明显下降。

    \subsubsection{Web Assembly运行环境}
        在设计上 wasm 代码被限制在沙盒环境下运行
        \cite{johnsonWaVeVerifiablySecure2023,WasmbpfStreamliningEBPF2024},
        这确保了其运行时安全,并让其成功地于浏览器之外为非网页程序提供沙盒环境
        \cite{narayanSwivelHardeningWebAssembly, WebAssemblySummaryOnSecurity, 9156135}。

    \subsubsection{Web Assembly内存模型}
        除此通过沙盒环境来隔离程序的安全机制以外,wasm还通过线性连续的栈内存来限制程序的访存操作。
        这一线性内存并不存储全局变量或者局部变量。全局变量保存在一张固定大小的、
        名为全局索引空间(Global index space)的表上;局部变量存储于一个受到保护的调用栈上,调用栈保存有函数的返回地址。

        线性内存在设计上不可执行,也不可跳转。然而wasm不允许内存被标记为只读,
        相反,所有线性内存上的数据均可写\cite{lehmannEverythingOldNew}。

        但由于栈上不存在类似于金丝雀的栈溢出检查标记,wasm有栈溢出风险。

    \subsubsection{Web Assembly数据类型}
        Wasm充分借鉴了安全语言设计,采用静态强类型系统,只提供四种基本数据类型
        \cite{wasmCommunityGroup,lehmannEverythingOldNew},
        分别为32位整数、32位浮点数、64位整数和64位浮点数。高级语言中的复杂类型,
        都在编译阶段被编译为这类基本类型。
    \subsubsection{Web Assembly控制流完整性}
        wasm控制流完整性检查,包括直接函数调用检查、函数返回检查
        以及间接函数调用检查三个方面\cite{WebAssemblySummaryOnSecurity,Daniel2019DiscoveringVI}。
        wasm的函数并不能执行任意地址跳转。函数地址被记录在一张表上,调用函数需要通过取出表上的索引才能完成跳转。

        间接函数跳转则是在函数需要在执行时动态确定(如C++的多态或函数指针赋值后调用的形式)
        才会被转译出来\cite{Daniel2019DiscoveringVI}。
        % Wasm 直接函数调用通过函数索引空间 [56] 中的索引来指定被调用函数并进行函数签名校验,
        % 如果函数签名不匹配,则该调用会触发校验异常并终止调用; 
        % 其次,Wasm 函数返回通过使用托管内存的调用栈,来保护函数返回地址;
        % 最后,Wasm间接函数调用在运行前对函数进行类型检查,保证基于类型的粗粒度的控制流完整性。
